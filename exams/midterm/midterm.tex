\documentclass[12pt]{exam}


\usepackage{cmap, type1ec}
\usepackage[T2A]{fontenc}
\usepackage[utf8]{inputenc}
\usepackage[russian]{babel}

\usepackage{hyperref}
\usepackage{alltt}

\usepackage[margin=1in]{geometry}
\usepackage{amsmath,amssymb}
\usepackage{multicol}
\usepackage{etoolbox}

\usepackage{tikz}
\usetikzlibrary{trees}
\usetikzlibrary{shapes.geometric}
\usetikzlibrary{positioning}

\newtoggle{first}
\toggletrue{first}
% \togglefalse{first}

\newcommand{\class}{Теория алгоритмов}
\newcommand{\term}{Весенний семестр 2018}
\newcommand{\examnum}{Контрольная.~Вариант \#\iftoggle{first}{1}{2}}
\newcommand{\examdate}{21 апреля 2017}
\newcommand{\timelimit}{\iftoggle{first}{12:10 -- 13:30}{16:40 -- 18:00}}

\pagestyle{head}
\firstpageheader{}{}{}
\runningheader{\class}{\examnum\ - Страница \thepage\ из \numpages}{\examdate}
\runningheadrule

\providecommand{\abs}[1]{\left\lvert{#1}\right\rvert}

\renewcommand{\solutiontitle}{}

\makeatletter
\newcommand{\iftoggleverb}[1]{%
  \ifcsdef{etb@tgl@#1}
    {\csname etb@tgl@#1\endcsname\iftrue\iffalse}
    {\etb@noglobal\etb@err@notoggle{#1}\iffalse}%
}
\makeatother


\begin{document}

\noindent
\begin{tabular*}{\textwidth}{l @{\extracolsep{\fill}} r @{\extracolsep{6pt}} l}
\textbf{\class} & \textbf{Студент:} & \makebox[3in]{\hrulefill}\\
\textbf{\term} &&\\
\textbf{\examnum} &&\\
\textbf{\examdate} \\
\textbf{\timelimit}
\end{tabular*}\\
\rule[2ex]{\textwidth}{2pt}%
\begin{center}
Оценки (заполняется преверяющими)\\
\addpoints
\gradetable[h][questions]
\end{center}
\noindent
\rule[2ex]{\textwidth}{2pt}

{\bf N.B.: кроме явных оговорок, каждая задача подразумевает наличие подробного решения и/или доказательства, предшествующего ответу. В противном случае задача не будет засчитана.}

\begin{questions}

\question[6] Поставьте {\bf \textcircled{+}} рядом с корректными и {\bf \textcircled{$-$}} рядом с некорректными утверждениями. Обосновывать свой выбор не требуется.
\begin{checkboxes}
    % \iftoggle{first}{%
      \CorrectChoice Если $f(n)=O(g(n))$ и $g(n)=O(h(n))$, то $h(n)=\Omega(f(n))$
      \begin{solution}
      Принадлежность к $O$ транзитивна, а $h(n)=\Omega(f(n))$ равносильно $f(n) = O(h(n))$.
      \end{solution}
    % }{%
    %   \CorrectChoice Если $f(n) = O(g(n))$ и $g(n) = O(h(n))$, то $h(n) = O(f(n))$
    %   \begin{solution}
    %   Принадлежность к $O$ транзитивна.
    %   \end{solution}
    % }

    \choice При заполнении хэш-таблицы выше заданного предела нужно выделить новый массив большего размера и скопировать туда элементы из старого используя ту же хэш-функцию.
    \begin{solution}
    Старая хэш-функция будет транслировать элементы лишь в префикс массива соответствующий его предыдущему размеру. Новая хэш-функция должна использовать массив целиком.
    \end{solution}

    % \iftoggle{first}{%
        \CorrectChoice Алгоритм с асимптотикой $O(n^2)$ может работать медленнее чем $O(n^3)$
        \begin{solution}
        Это возможно при определённом сочетании постоянных множителей в функции $T(n)$ и размера данных.
        \end{solution}

    % }{%
    %     \choice Алгоритм с асимптотикой $O(n^2)$ всегда работает медленнее чем $O(\log n)$
    %     \begin{solution}
    %     Это может не выполняться при определённом сочетании постоянных множителей в функции $T(n)$ и размера данных.
    %     \end{solution}
    % }

    % \iftoggle{first}{%
        \choice В {\em max-heap} операции добавления, удаления максимума, поиска максимума и поиска минимума выполняются за $O(\log n)$.
        \begin{solution}
        Минимальный элемент в max-heap находится в одном из листовых элементов дерева, которых порядка $n/2$. Значит на его поиск уйдёт $O(n)$
        \end{solution}
    % }{%
    %     \CorrectChoice Массив {\tt [20, 15, 18, 7, 9, 5, 12, 3, 6, 2]} является {\em max-heap}.
    %     \begin{solution}
    %     Действительно:
    %     {\tt Потомки A[1] = 20 : A[2] = 15 < 20,  A[3] = 18 < 20.}\\
    %     {\tt Потомки A[2] = 15 : A[4] = 7 < 15,  A[5] = 9 < 15.}\\
    %     {\tt Потомки A[3] = 18 : A[6] = 5 < 18,  A[7] = 12 < 18.}\\
    %     {\tt Потомки A[4] = 7  ~: A[8] = 3 < 7,  A[9] = 6 < 7.}\\
    %     {\tt Потомок A[5] = 9  ~: A[10] = 2 < 9}\\
    %     {\tt A[6]...A[10] не имеют потомков}\\
    %     \end{solution}
    % }

    \choice При наличии указателей на начало и конец односвязного списка операция удаления последнего элемента занимает $O(1)$.
    \begin{solution}
    Для удаления последнего элемента необходимо присвоить {\tt None} предпоследнему. Ссылка на конец не помогает его найти: до него всё равно нужно проитерироваться от начала списка. Отсюда сложность удаления с конца $O(n)$.
    \end{solution}

    \choice В динамическом массиве удаление 4-го с конца элемента выполнимо за $\Omega(n)$.
    \begin{solution}
    Для его удаления нужно перенести три последних элемента на одну позицию к началу массива. С учётом зафиксированного числа элементов, эту операцию можно считать $O(1)$.
    \end{solution}
\end{checkboxes}


\question
Наибольший (или <<первый по величине>>) элемент {\em max-heap} всегда располагается по индексу $0$. Пусть имеется {\em max-heap} размера 15 без повторяющихся элементов. Перечислите список индексов, на которых {\em может} находиться элемент:
\begin{parts}
\part[2] второй по величине; 
\part[3] третий по величине.
\part[1] Также, напишите функцию, которая принимает массив {\em max-heap} и возвращает {\em второй по величине} элемент. Можно считать, что в массиве не менее 2-х элементов, а значит такой элемент всегда найдётся.
\end{parts}

\begin{solution}
............
\end{solution}

\question[5] Дана коллекция $S$ содержащая $n$ целых чисел из диапазона $[0, k]$. Подробно опишите построение структуры данных, с помощью которой за время $O(1)$ для заданных $a, b \in S : a < b$ можно найти, сколько чисел из $S$ попадают в диапазон $[a, b]$.

{\em (Ограничений на $S$ не накладывается; в частности, числа в ней могут повторяться.)}

\begin{solution}
Следует использовать вычисление частичных сумм из сортировки подсчётом.
\end{solution}


\question[3] Дана пустая хэш-таблица размера 10 использующая открытую адресацию со стратегией linear probing. Используется хэш-функция $h(k) = k \mod 10$. После вставки шести элементов хэш-таблица выглядит так:\\
{\tt [None, None, 42, 23, 34, 52, 46, 33, None, None]}\\
Выпишите элементы в том порядке, в котором их могли добавлять в эту хэш-таблицу.

\question[6] Докажите или опровергните, что время асимптотическое время работы этой функции составляет $O(n^2)$.
\begin{alltt}
def foo(n):
    # tuple хэшируется, в отличие от list
    seq = tuple(range(0, n))
    accum = \{\}
    for i in range(n):
        for j in range(i+1, n):
            subseq = seq[i:j]  # формируем новый tuple
            accum[subseq] = (i, j)
\end{alltt}

\question[8] На семинарах мы разбирали {\em Dutch national flag problem}: сортировку массива, элементы которого принимают всего три значения. Алгоритм базировался на quicksort и работал за линейное время. В этом задании нужно реализовать такой же алгоритм, но вместо массива на вход подаётся односвязный список.

Следует считать, что элементы односвязного списка могут принимать лишь значения 0, 1 или 2. Функция {\tt three\_quick\_sort} должна возвращать голову отсортированного списка. Сортировка должна производиться in-place и за линейное время.

\newpage
\begin{alltt}
class Node:
    
    def __init__(self, value: int, next_=None):
        self.value = value
        self.next = next_


# Пример входных данных:
sample = Node(0, Node(1, Node(0, Node(2, Node(2, Node(2, Node(0, Node 1)))))))

        
def partition(pivot: int, lst: Node) -> Node:
    ...
    
def three_quick_sort(lst: Node) -> Node:
    ...
\end{alltt}

\question[6] Функция ниже прибавляет единицу к двоичному числу, записанному в виде массива бит. В этой записи младший разряд имеет индекс 0. Например, {\tt [1, 1, 0, 1]} (число $1011_2$) функция преобразует в {\tt [0, 0, 1, 1]} (число $1100_2$).

С помощью метода <<бухгалтерского учёта>> докажите, что функция имеет амортизационную сложность $O(1)$.

\begin{alltt}
def increment(number: List[int]):
    i = 0
    while i < len(number) and number[i] == 1:
        number[i] = 0
        i = i + 1
        
    if i < len(number):
        number[i] = 1
\end{alltt}



\end{questions}

\end{document}