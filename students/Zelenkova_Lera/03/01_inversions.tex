\documentclass[12pt]{extarticle}
\usepackage{geometry,nopageno}
\geometry{a5paper,left=1cm,right=1cm,top=1cm,bottom=1cm}
\usepackage{cmap, type1ec}
\usepackage[T2A]{fontenc}
\usepackage[utf8]{inputenc}
\usepackage[russian]{babel}

\usepackage{verbatim,nameref}
\usepackage{amsmath,amsthm,amstext,amssymb,amscd,
            mathtools,mathrsfs,dsfont}
            
\newtheorem*{problem}{Задача}

\begin{document}

\begin{problem}[1]
{\em Инверсией} в перестановке $\pi$ называется пара индексов $(i,j)$ такая что $i<j$, но $\pi_i>\pi_j$. Докажите, что в перестановке из $n$ элементов может быть не более $n(n-1)/2$ инверсий. В какой перестановке количество инверсий ровно $n(n-1)/2$?
\end{problem}

\begin{proof}

{\em Максимальное количество инверсий - в массиве, упорядоченном по убыванию. Например, [5, 4, 3, 2, 1]. Можно посчитать там кол-во перестановок вручную слева направа: 4+3+2+1 = 10. И по формуле: $(5*4)/2 = 10$. Большего количества инверсий быть не может, так как упорядоченный по убыванию массив - это единственный вариант, где слева направа каждый предыдущий элемент больше последующего, если индекс предыдущего элемента < индекса последующего.В неупорядоченном массиве (даже если там только один элемент неупорядочен), например, [1, 5, 4, 3, 2], неупорядоченный элемент уменьшает кол-во инверсий, так как найдется хотя бы один элемент с бОльшим индексом, но меньшим значением.}

\end{proof}

\end{document}
