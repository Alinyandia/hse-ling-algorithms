\documentclass[12pt,a4paper]{report}
% \textheight = 30cm
% \voffset = -36pt
\footskip = 0cm

\usepackage{cmap}
\usepackage{type1ec}
\usepackage[T2A]{fontenc}
\usepackage[utf8]{inputenc}
\usepackage[russian]{babel}

\usepackage{amsmath,amstext,amssymb}
\usepackage{fullpage}

\usepackage{enumitem}
\usepackage{ifpdf}
\ifpdf
  \usepackage[pdftex]{graphicx}
  \usepackage[pdftex,unicode,bookmarks=false]{hyperref}

  \pdfminorversion=5
  \pdfcompresslevel=9
  \pdfobjcompresslevel=9
\fi

\pagestyle{empty}


% Константа
\def\const{\mathop{\mathrm{const}}\nolimits}

\renewcommand{\thesection}{\arabic{section}}
\renewcommand{\thesubsection}{}

\begin{document}

\begin{center}
\textbf{\large{Теория алгоритмов. Листок 1}}\\
BSc, компьютерная лингвистика, НИУ ВШЭ\\
Выдан: 23 января 2018\\
\end{center}


\begin{enumerate}
  \item\,[1] Оцените время работы функции в худшем случае в терминах $O(f(n))$
  \begin{verbatim}
  def foo(n):
    r = 0
    for i in range(1, n):
      for j in range(i+1, n)
        for k in range(1, j):
          r += 1
    return(r)\end{verbatim}%

  \item\,[1] Можно ли в общем случае считать, что код ниже выполняется за время $O(|A|)$, где $|A|$ -- количество чисел в массиве {\tt A}? Ответ поясните.
  \begin{verbatim}
  A: List[int] = [ ... ]
  r = 1
  for e in A:
    r  = r * e\end{verbatim}%
  \item\,[1] Приведите пример функции, для которой не существует асимптотической оценки через $\Theta$.
  \item\,[1] Пользуясь только определением $\Theta$ (т.е. не вычисляя предел) докажите, что при $a_d > 0$ и $d=\const$ выполняется
  $$
  \sum_{i=0}^{d}a_i n^i = \Theta(n^d)
  $$
  \item\,[2] Пользуясь формулой Стирлинга, докажите, что $c^n = O(n!)$.
  \item\,[2] Докажите или опровергните:
    \begin{enumerate}[label=(\alph*)]
      \item $\log{n} + \sqrt{n\sqrt{n}} \in \Theta(\log{n})$
      \item $f(n) + O(f(n)) = \Theta(f(n))$
    \end{enumerate}

  \item\,[2] Расположите функции <<по возрастанию>> в асимптотическом смысле, т.е. в последовательность $f_i$ так, чтобы $\lim_{n\to\infty} f_i(n)/f_j(n) = 0$ при $i < j$

  $$
n^2\log_2n ~~~~~ n(\log_2n)^2 ~~~~~ \sum_{i=0}^n 2^i ~~~~~ \log_2\left(\sum_{i=0}^n 2^i\right)
  $$



\end{enumerate}

\end{document}