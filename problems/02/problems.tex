\documentclass[12pt,a4paper]{report}
% \textheight = 30cm
% \voffset = -36pt
\footskip = 0cm

\usepackage{cmap}
\usepackage{type1ec}
\usepackage[T2A]{fontenc}
\usepackage[utf8]{inputenc}
\usepackage[russian]{babel}

\usepackage{amsmath,amstext,amssymb}
\usepackage{fullpage}

\usepackage{enumitem}
\usepackage{ifpdf}
\ifpdf
  \usepackage[pdftex]{graphicx}
  \usepackage[pdftex,unicode,bookmarks=false]{hyperref}

  \pdfminorversion=5
  \pdfcompresslevel=9
  \pdfobjcompresslevel=9
\fi

\pagestyle{empty}


% Константа
\def\const{\mathop{\mathrm{const}}\nolimits}

\renewcommand{\thesection}{\arabic{section}}
\renewcommand{\thesubsection}{}

\begin{document}

\begin{center}
\textbf{\large{Теория алгоритмов. Листок 2}}\\
BSc, компьютерная лингвистика, НИУ ВШЭ\\
Выдан: 20 февраля 2018\\
\end{center}

\begin{enumerate}
  \item\,[{\tt vector\_copy}] Рассмотрим динамический массив, размер которого увеличивается в $\alpha$ раз при добавлении очередного элемента в целиком заполненный массив. Для $\alpha=2$ посчитайте, сколько операций копирования совершается при добавлении $n$ элементов в изначально пустой массив.

  {\it (Одной операцией считается копирование одного элемента массива. В задаче предполагается явный подсчёт, без использования <<метода бухучёта>> и асимптотических оценок.)}

  \item\,[{\tt vector\_alpha}] Покажите, что оценка $O(n)$ общего времени выполнения $n$ операций вставки в динамический массив остаётся справедливой для любого другого $\alpha > 1$.

  \item\,[{\tt two\_stacks\_amortized}] На занятиях рассматривалась реализация очереди с помощью двух стеков. Используя <<метода бухучёта>>, докажите амортизированную временную сложность операций {\tt pop} и {\tt push} в такой очереди.

  \item\,[{\tt stacks\_array}] Опишите реализацию {\em двух} стеков с помощью {\em одного} массива размера $n$. Ни один из стеков не должен переполняться, если общее число элементов в стеках не превышает $n$. Операции {\tt pop} и {\tt push} должны гарантированно выполняться за $O(1)$.
\end{enumerate}

\end{document}