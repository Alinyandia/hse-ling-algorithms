\chapter{Syllabus}

В прошлом году лекции и семинары практически не отличались, в этом попробуем решать больше задач на семинарах.

\section*{Оценка}

$$
G_{final} = 0.3 \cdot G_{hw} + 0.3 \cdot G_{test} + 0.4 \cdot G_{exam}
$$

Домашние задания -- после семинаров. Контрольная -- между модулями.

\section*{Примерный план}

1-й модуль:

\begin{itemize}
  \item Анализ сложности алгоритмов
  \item Алгоритмы сортировки
  \item Структуры данных: списки, деревья, хэш-таблицы, кучи, системы непересекающихся множеств
\end{itemize}

2-й модуль:

\begin{itemize}
  \item Жадное программирование. Динамическое программирование.
  \item Алгоритмы на графах.
  \item Алгоритмы на строках.
  \item Что-нибудь специфичное для лингвистики/численных методов/баз данных
\end{itemize}

\section*{Литература}

\begin{enumerate}
  \item {\bf T.~Cormen et al. {\em Introduction to Algorithms}}
  \item Steven Skiena {\em The Algorithm Design Manual}
  \item Бабенко М.А.,~Левин М.В. {\em Введение в теорию алгоритмов и структур данных}
\end{enumerate}
